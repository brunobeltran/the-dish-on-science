%%%%%%%%%%%%%%%%%%%%%%%%%%%%%%%%%%%%%%%%%%%%%%%%%%%%%%%%%%%%%%%%%%%%%%
% Instructions for Formatting and Posting Articles to
% thedishonscience.com
%
%%%%%%%%%%%%%%%%%%%%%%%%%%%%%%%%%%%%%%%%%%%%%%%%%%%%%%%%%%%%%%%%%%%%%%
% Edit the title below to update the display in My Documents
\title{How to Post}
%
%%% Preamble
\documentclass[paper=a4, fontsize=11pt]{scrartcl}
\usepackage[T1]{fontenc}
\usepackage{fourier}

\usepackage[english]{babel}                                                            % English language/hyphenation
\usepackage[protrusion=true,expansion=true]{microtype}
\usepackage{amsmath,amsfonts,amsthm} % Math packages
\usepackage[pdftex]{graphicx}
\usepackage{hyperref}
\usepackage{forest}
\usepackage{float}


%%% Custom sectioning
\usepackage{sectsty}
\allsectionsfont{\centering \normalfont\scshape}


%%% Custom headers/footers (fancyhdr package)
\usepackage{fancyhdr}
\pagestyle{fancyplain}
\fancyhead{}                                            % No page header
\fancyfoot[L]{}                                            % Empty
\fancyfoot[C]{}                                            % Empty
\fancyfoot[R]{\thepage}                                    % Pagenumbering
\renewcommand{\headrulewidth}{0pt}            % Remove header underlines
\renewcommand{\footrulewidth}{0pt}                % Remove footer underlines
\setlength{\headheight}{13.6pt}


%%% Equation and float numbering
\numberwithin{equation}{section}        % Equationnumbering: section.eq#
\numberwithin{figure}{section}            % Figurenumbering: section.fig#
\numberwithin{table}{section}                % Tablenumbering: section.tab#
\newcommand{\dishurlplain}[1]{http://www.brunobeltran.org/#1}
\newcommand{\dishurl}[1]{\url{\dishurlplain{#1}}}


%%% Maketitle metadata
\newcommand{\horrule}[1]{\rule{\linewidth}{#1}}     % Horizontal rule

\title{\
        %\vspace{-1in}
        \usefont{OT1}{bch}{b}{n}
        % \normalfont{}\normalsize \textsc{School of random department names} \\ [25pt]
        \horrule{0.5pt} \\[0.4cm]
        \huge How to Post on \\ \dishurl{} \\
        \horrule{2pt} \\[0.5cm]
}
\author{\
        \normalfont{}                     \normalsize
        Bruno Beltran\\[-3pt]             \normalsize
        \today
}
\date{}


%%% Begin document
\begin{document}
\maketitle

\section{Important Notice}
For now, the website is being hosted at \dishurl{}. Everywhere in this document where this
URL is mentioned, it will be changed whenever we get our final URL, whatever
that may be (probably to something like
\url{http://dishonscience.stanford.edu/}).

\section{What Can Go in My Article?}
The Dish on Science is a Stanford graduate student blogging group.
As a Stanford graduate student group, we must adhere to Stanford's
\href{https://adminguide.stanford.edu/chapter-6/subchapter-2/policy-6-2-1}{``Computer
and Network Usage Policy''}. In short, this means that in your official writing
for the Dish:\@
\begin{enumerate}
    \item DO NOT expose anyone's online identity.
    \item DO NOT use The Dish to advocate for yourself or a business you know,
        even in passing.
    \item DO NOT explicitly advocate for any political group.
    \item DO NOT violate copyright law (see
        Section~\ref{sec:copyright} below for details).
\end{enumerate}

As long as you keep to the simplified guidelines above and maintain a
professional tone, we should not have any problems. If in doubt, however,
consult the full university policy at
\url{https://adminguide.stanford.edu/chapter-6/subchapter-2/policy-6-2-1}.

\section{Copyright Law}\label{sec:copyright}

This is fairly easy to summarize. If you didn't make it, and it is not both from
a credible source and explicitly marked as Creative Commons (for reuse) or
public domain, then you have to ask the person that owns it for permission to
use it.

In particular, this means that if you want to reproduce a figure from a journal,
you invariably \textbf{must} ask permission from the copyright holder. This is
usually the journal, but you can find out for sure by searching either the PDF
or the webpage of the article for a Copyright notice for the article and seeing
who the copyright is assigned to. Do not mistake the article being available
from a separate site as meaning that it is okay to steal pictures from it. Many
good, well-meaning blogs have been sued for exactly this practice.

As an important aside, realize that this document is not legally binding in any way and its author
is not a lawyer, so please use your own best judgement as necessary to obey all
relevant laws and regulations.

\section{How to Format a Post}
The Dish's website is maintained by the Website Administrator, and the Editor in
Chief is in charge of managing timely submission of articles to the site, per
their duties as outlined in
\href{\dishurlplain{documents/dish-constitution.pdf}}{the constitution}. A
large part of the uploading is automated to both ensure consistency of design
across the site and to minimize the extra work that has to be done by the
Editor.  As such, we ask that all posts follow the following format
\textit{exactly}.

A minimal, correct example article can be found at
\dishurl{documents/minimal-example-post}. A post that leverages all optional
features of the website available without special requests can be found at
\dishurl{documents/maximal-example-post}. In places where this document is
ambiguous, these examples should serve as an official reference.

For an article with desired URL
\dishurl{posts/post-name-url}, the following folder structure is required.
\begin{figure}[H]
\begin{forest}
  for tree={%
    font=\ttfamily,
    grow'=0,
    child anchor=west,
    parent anchor=south,
    anchor=west,
    calign=first,
    edge path={%
      \noexpand\path{} [draw, \forestoption{edge}]
      (!u.south west) + (7.5pt,0) |- node[fill,inner sep=1.25pt] {} (.child anchor)\forestoption{edge label};
    },
    before typesetting nodes={%
      if n=1
        {insert before={[,phantom]}}
        {}
    },
    fit=band,
    before computing xy={l=15pt},
  }
[post-url-name
  [images
    [image1.png]
    [image2.svg]
    [\ldots{}]
  ]
  [post.md]
  [post\_info.json ]
]
\end{forest}
\caption{Post folder structure guidelines.}\label{fig:folder-structure}
\end{figure}

The \texttt{post\_info.json} and \texttt{post.md} files must have
\textbf{exactly those names}. The image files can be called anything, as long as
they're correctly linked to in the article, but the
folder containing them must be called \texttt{images}. See Section~\ref{sec:image-links}
for how to correctly link to your images.

\subsection{How to Make \texttt{\textbf{post\_info.json}}}
Please see the examples mentioned above for how to format the file. If what
exactly should be changed or updated is not clear to you, the file is in the
very popular \href{http://www.xul.fr/ajax-javascript-json.html}{JSON} file
format, which is well explained in various places on the web, such as
\url{http://www.xul.fr/ajax-javascript-json.html}.

In what follows, all image links should be made relative to the article
directory. More explicitly, please use
\begin{verbatim}"./images/image-file-name.png"\end{verbatim}
filling in the correct filename and extension (\texttt{.svg}, \texttt{.jpg},
etc.) each time a link to an image file is required.

\noindent{}The required fields are:\@
\begin{itemize}
    \item \emph{Post Title}: Try to keep below 200 characters.
    \item \emph{Main Post Image}: A link to the main image that will be
        displayed alongside links to the article and at the top of the article.
    \item \emph{Post ``Blurb''}: A short, sub-sentence-long description of what
        the article is about. If you want your article to have a subtitle, this
        is the most appropriate place.
    \item \emph{Post Description}: A short paragraph that will be displayed
        beneath your article's image when listing articles. This should be your
        call to action, and after reading this, a visitor to the site should
        want to click on your article to read more. The text ``(Click to read
        more\ldots{})'' will be automatically included after this description, so do
        not include it yourself.
    \item \emph{Team Name}: The official name of the team or teams that have edited the
        post.
\end{itemize}

\noindent{}The following (optional) fields are also allowed.
\begin{itemize}
    \item \emph{Author name or names}: Please use your legal name to make this consistent. If
        you submit one article as ``Bob Caldwell'' and another as ``Bobby
        Caldwell'', there will be no way of assigning these to the same person.
        If this field is omitted, the group name will be used as a stand in for
        the author name.
    \item \emph{Author nickname or nicknames}: One per author. This will be the name that is
        actually displayed on the site next to your (optional) picture.
    \item \emph{Author headshot or headshots}: This will be a link to the picture that will
        display next to your name. You can either provide an image and link to
        that file name inside the ``\texttt{images}'' folder, or use
        ``\texttt{/images/headshot.png}'' for a picture of a cute hedgehog instead.

        You can always check what the name of the file you provided is if you've
        forgotten by looking for yourself at ``\dishurl{images/}''.
        For example, after finding my picture at
        ``\dishurl{/images/bruno-beltran-2015.png}'', I would use
        ``\texttt{/images/bruno-beltran-2015.png}''.
    \item \emph{Squared Post Image}: A link to a cropped version of the post image that
        can be used as a thumbnail for the post. If this is not included, a
        square thumbnail will automagically be generated from the ``Main Post Image''.
\end{itemize}


\subsection{How to Make \texttt{\textbf{post.md}}}

\subsubsection{Basic Markdown}

Each group is free to (and should) use whatever format (e.g.\ Word, plain text,
LaTeX+git) makes it easiest for your groups to get the articles written and
reviewed. When it comes time for you to submit, however, the process of making
the article look like you want it to on the webpage will require that you use a web-friendly
format. In order to minimize the amount of work this will take both for the
authors and the editor, we will use ``Markdown'' syntax to specify the article's
formatting. Don't be afraid, this does not involve programming, learning a new
language, or knowing how to cartwheel. The syntax should feel familiar to anyone
that has posted in a forum, on reddit, on Github, or on other website that allows
formatting comments/posts.

The best way to explain how it works is to simply point you to a cheat
sheet that shows you how to do everything you could possibly want,
from italics and bolding to tables, links and images:

\url{https://github.com/adam-p/markdown-here/wiki/Markdown-Cheatsheet}

Here's a site that lets you type in markdown and shows you what it would
look like on the web, with some built in examples.

\url{https://stackedit.io/editor}

In practice, the only thing to keep in mind is that you'll need to put
two newlines (i.e.\ hit enter twice) every time you want to start a new
paragraph, and everything else can be looked up easily on the
\href{https://github.com/adam-p/markdown-here/wiki/Markdown-Cheatsheet}{cheatsheet}.

Various people have found
that a relatively efficient way to get the article formatted is to just copy/paste
from word into \href{https://stackedit.io/editor}{StackEdit} and fix the few problems that appear. More advanced users
might want to try using an automatic conversion tool like
\href{http://pandoc.org/demos.html}{Pandoc} or the
\href{http://www.writage.com/}{``Writage''} plugin.

\subsubsection{Images}\label{sec:image-links}
In order to get an image in your article, simply include it in the
\texttt{``images''} subfolder of your post as demonstrated by
Figure~\ref{fig:folder-structure}. Then, the usual syntax for including a
picture in a Markdown document should work using the relative file path
(i.e.\ starting with ``\texttt{./images/}'').

For example, to create a link to ``\texttt{image1.png}'' in the example in
Figure~\ref{fig:folder-structure}, one would, inside of \texttt{post.md}, use
the syntax

\begin{verbatim}![alt text](./images/image1.png "hover text")\end{verbatim}

The text \texttt{alt text} will then appear as a stand-in if the image fails to
load or loads too slowly, and the text \texttt{hover text} will appear if the
reader hovers their mouse over the image.

When composing the article, it might be helpful to use an external tool like
\href{https://stackedit.io/editor}{StackEdit} to view your article as you type.
If you want to also see your images to get a rough idea of how they will appear
in the article, you have to provide a valid link to the file such that the
program that you're using can find the image. For example, when using
\href{https://stackedit.io/editor}{StackEdit}, the easiest solution is to upload
the images to an online file sharing program like Dropbox and use the dropbox
link instead of \texttt{./images/image1.png} in the example above while you're
writing the article. Just make sure that the version of the Markdown file that you
submit to the editor has the links formatted as in the example above.

Finally, remember that on the web, portable formats are king. Vectorized graphics are
especially nice to have. In rough order of decreasing preference, please use one
SVG, PNG, or JPG/JPEG formatted files. If you have a graphic in another format,
for example from illustrator/inkscape, photoshop/gimp, or powerpoint (shame on
you!) then please convert it to one of the three above formats unless you
understand the implications.


%%% End document
\end{document}
